\chapter{Negation: Declarative int.}

\paragraph{Disclaimer.} Some notions already discussed in the Logic course will be first omitted. If time will be available, I will add them.

\section{Definitions}

% \begin{dfn}
%     Given a program \prog, a query \(Q\) and a substitution \(\theta\),
%     \begin{itemize}
%         \item \(\restr{\theta}{\Var(Q)}\) is a \emph{correct answer substitution} of \(Q\) iff \(\comp(\prog) \models Q\theta\),
%         \item \(Q\theta\) is a correct instance of \(Q\) iff \(\comp(\prog) \models Q\theta\).
%     \end{itemize}
% \end{dfn}

\begin{dfn}
    \label{decl-neg:dfn-1}
    The \emph{dependency graph} \(D_\prog\) of an extended program \prog is a directed graph with labeled edges, where:
    \begin{itemize}
        \item the nodes are the predicate symbols of \prog,
        \item the edges are either labeled by \(+\) (positive edge) or \(-\) (negative edge);
        \item \(D_\prog\) has an edge \(p \edplus q\) if \prog contains \(p(s_1,\dotsc,s_m) \from \mathbf{L}, q(t_1,\dotsc,t_n),\mathbf{N}\);
        \item \(D_\prog\) has an edge \(p \edminus q\) if \prog contains \(p(s_1,\dotsc,s_m) \from \mathbf{L}, \neg q(t_1,\dotsc,t_n),\mathbf{N}\);
    \end{itemize}
    Given \(p\) and \(q\) predicate symbols, \(p\) \emph{depends evenly} (resp. \emph{oddly}) on \(q\) if there is a path in \(D_\prog\) from \(p\) to \(q\) with an even (resp. odd) number of negative edges.

    Given a program \prog and an extended query \(Q\), the program is:
    \begin{itemize}
        \item \emph{strict} wrt \(Q\), if no predicate symbol occurring in \(Q\) depends both evenly and oddly on a predicate symbol in the head of a clause in \prog;
        \item \emph{hierarchical}, if \(D_\prog\) is acyclic;
        \item \emph{stratified}, if no cycle containing a negative edge appear in \(D_\prog\).
    \end{itemize}
\end{dfn}

\begin{rem}
    As a conseguence of~\cref{decl-neg:dfn-1}, \(p\) always depends evenly on \(p\), since a \(0\)-path has even length.
\end{rem}

\begin{dfn}[fairness]
    \label{decl-neg:dfn-2}
    An extended selection rule \rulesel is \emph{fair} if, for every \gls{sldnf}-tree \(F\) via \rulesel and for every branch \(\xi\) in \(F\), either \(\xi\) is failed or for every literal \(L\) occurring in a query of \(\xi\), some further instantiated version of \(L\) is selected within a finite number of derivation steps.
\end{dfn}

\begin{exa}
    The selection rule ``select the leftmost literal'' is unfair, according to~\cref{decl-neg:dfn-2}.
    On the other hand, ``select leftmost literal to the right of the literals introduced at the previous derivation step, if it exists, otherwise, select leftmost literal'' is a fair selection rule.
\end{exa}

\begin{dfn}
    The extended \emph{consequence operator} of a program \prog, given a Herbrand interpretation \(I\), is
    \begin{equation*}
        T_\prog(I) := \lbrace H \mid H \from \mathbf{B} \in \grnd(\prog), I \models \mathbf{B} \rbrace.
    \end{equation*}
\end{dfn}

\begin{rem}
    If \prog is a definite program, the consequence operator has properties previously seen.
    But now, all these nice properties are lost!
\end{rem}

\begin{dfn}[Standardized Herbrand interpretation]
    Let \(=\) be a binary predicate symbol, standing for \emph{equality}, not in \gls{preds} and let \(I\) be an interpretation for \gls{funcs} and \gls{preds}.
    Then,
    \begin{equation*}
        I_= := I \cup \lbrace t = t \mid t \in \gls{HU}\rbrace
    \end{equation*}
    is called a \emph{standardized} Herbrand interpretation for \gls{funcs} and \(\gls{preds} \cup \lbrace = \rbrace\).
\end{dfn}

\begin{dfn}[Supported Herbrand interpretation]
    A Herbrand interpretation \(I\) is \emph{supported} iff for every \(H \in I\) there is some ground clause \(H \from \mathbf{B}\) such that \(I \models \mathbf{B}\).
\end{dfn}

\section{Algorithms}

\paragraph{Completion.}
The \emph{completion} of an extended program \prog, denoted by \(\comp(\prog)\), is the set of formulas constructed from \prog by applying the following steps:
\begin{enumerate}
    \item Associate with every \(n\)-ary predicate symbol \(p\) a sequence of pair-wise distinct variables \(x_1,\dotsc,x_n\) not occurring in \prog;
    \item transform each clause \(c \colon p(t_1,\dotsc,t_n) \from \mathbf{B}\) into
    \begin{equation*}
        p(x_1,\dotsc,x_n) \from
        x_1 = t_1,\dotsc,x_n = t_n, \mathfb{B}
    \end{equation*}
        \item transform each resulting formula \(p(x_1,\dotsc,x_n) \from G\) into
        \begin{equation*}
            p(x_1,\dotsc,x_n) \from \exists{\mathbf{z}}\,G,
        \end{equation*}
        where \(\mathbf{z}\) is a sequence of elements of \(\Var(c)\);
        \item for every \(n\)-ary predicate symbol \(p\), let
        \begin{equation*}
            p(x_1,\dotsc,x_n) \from \exists{\mathbf{z}_i} G_i \qquad i = \oto{m}, m \ge 0
        \end{equation*}
        be all the implications obtained in the previous step:
        \begin{itemize}
            \item if \(m \ge 0\) replace the implications by the formula
            \begin{equation*}
                \forall{x_1,\dotsc,x_n} p(x_1,\dotsc,x_n) \leftrightarrow \exists{\mathbf{z}_1} G_1 \lor \dotsb \lor \exists{\mathbf{z}_m} G_m
            \end{equation*}
            \item otherwise, add
            \begin{equation*}
                \forall{x_1,\dotsc,x_n} p(x_1,\dotsc,x_n) \leftrightarrow \bot
            \end{equation*}
        \end{itemize}
        \item add the standard axioms of equality: reflexivity, symmetry, transitivity, \(f\)-substitutivity, \(p\)-substitutivity;
        \item add the standard axioms of inequality:
        \begin{itemize}
            \item if for some \(i = \ito{n}\), \(x_i \ne y_i\), then \(f(x_1,\dotsc,x_n) \ne f(y_1,\dotsc,y_n)\);
            \item if \(f \ne g\), then \(f(x_1,\dotsc,x_n) \ne g(y_1,\dotsc,y_m)\);
            \item if \(x\) is a proper subterm of \(t\), then \( \ne t\).
        \end{itemize}
    \end{enumerate}

\section{Theorems \& friends}

\subsection{Soundness of \gls{sldnf}-resolution}

\begin{thm}
    \label{dec-neg:thm-1}
    If there exists a successful \gls{sldnf}-derivation of \(\prog \cup \lbrace Q \rbrace\) with \gls{cas} \(\theta\), then \(\comp(\prog) \models Q\theta\).
\end{thm}

\begin{cor}[of~\cref{dec-neg:thm-1}]
    If there exists a successful \gls{sldnf}-derivation of \(\prog \cup \lbrace Q \rbrace\), then \(\comp(\prog) \models Q\theta\).
\end{cor}

\begin{rem}
    In the general case, \gls{sldnf}-resolution is \emph{not complete}!
    Several examples on the slides are meant ot show that\ldots
\end{rem}

\subsection{``Completeness'' of \gls{sldnf}-resolution}

\begin{thm}[Restricted completeness]
    \label{dec-neg:thm-2}
    Let \prog be a hierarchical and allowed program, \(Q\) an allowed query.
    If \(\comp(\prog) \models Q\theta\) for some \(\theta\) such that \(Q\theta\) is ground, then there exists  a successful \gls{sldnf}-derivation of \(\prog \cup \lbrace Q \rbrace\) with \gls{cas} \(\theta\).
\end{thm}

\begin{rem}
    Theorem~\ref{dec-neg:thm-2} holds, if the selection rule is \emph{safe}. Otherwise, it may not hold.
\end{rem}

\begin{thm}
    \label{de-neg:thm-3}
    Let \prog be a hierarchical and allowed program, \(Q\) an allowed query, such that \prog is strict wrt \(Q\).
    If \(\comp(\prog) \models Q\theta\) for some \(\theta\) such that \(Q\theta\) is ground, then there exists  a successful \gls{sldnf}-derivation of \(\prog \cup \lbrace Q \rbrace\) with \gls{cas} \(\theta\).
\end{thm}

\begin{rem}
    Theorem~\ref{dec-neg:thm-3} holds, if the selection rule is \emph{safe} and \emph{fair}. Otherwise, it may not hold.
\end{rem}

\begin{lem}
    Let \prog be an extended program and \(I\) a Herbrand interpretation.
    Then, \(I \models \prog\) iff \(T_\prog(I) \subseteq I\).
\end{lem}
\begin{proof}
    First, we know that \(I \models \prog\) iff \(I \models \mathbf{B}\) implies \(I \models H\) for all the ground clauses \(H \from \mathbf{B}\) in \(\grnd(\prog)\).
    Since \(I\) is a Herbrand interpretation, \(I \models H\) is equivalent to \(H \in I\).
    By definition of \(T_\prog(I)\), if \(I \models \mathbf{B}\), then \(H \in T_\prog(I)\).
    Thus, we found that if \(H \in T_\prog(I)\), then \(H \in I\), hence \(T_\prog(I)\).

    Going backwards yields the other side of the proof.
\end{proof}

\begin{lem}
    \label{lem:decl-neg-supp}
    Let \prog be an extended program and \(I\) a Herbrand interpretation. Then, \(I_= \models \comp(\prog)\) if and only if \(T_\prog(I) = I\).
\end{lem}
\begin{proof}
    First, notice that \(I_=\) is a model for the standard axioms of equality and inequality.

    \(I_= \models \comp(\prog)\) is equivalent to
    \begin{equation*}
        I \models \left(H \leftrightarrow \bigvee_{(H \from \mathbf{B}) \in \grnd(\prog)} \mathbf{B}\right) \qquad \text{for each ground atom \(H\).}
    \end{equation*}
    This, in turn, is equivalent to say that, for each ground atom \(H\), \(H \in I\) if and only if \(I \models \mathbf{B}\) for some \(H \from \mathbf{B} \in \grnd(\prog)\).
    This means, by definition of \(T_\prog\), that \(H \in I\) if and only if \(H \in T_\prog(I)\).
    Therefore, we conclude that \(I_= = T_\prog(I)\).
\end{proof}

\begin{lem}
    Let \prog be an extended program and \(I\) a Herbrand interpretation. Then, \(T_\prog(I) = I\) iff \(I \models \prog\) and \(I\) is supported.
\end{lem}
\begin{proof}
    First, define \(\mathcal{B}(H) := \lbrace \mathbf{B} \mid H \from \mathbf{B} \in \grnd(P) \rbrace\).
    Assume that \(I \models \prog\) and \(I\) is supported.
    This is equivalent to say that, for every \(\mathbf{B} \in \mathcal{B}(H)\), \(I \models \mathbf{B}\) implies \(I \models H\) (since \(I \models \prog)\) and for every \(H \in I\), \(I \models \bigvee_{\mathbf{B} \in \mathcal{B}(H)} \mathbf{B}\), since \(I\) is supported.
    This holds iff for every ground atom \(H\), \(I \models H \from \bigvee_{\mathbf{B} \in \mathcal{B}(H)} \mathbf{B}\) and \(I \models H \to \bigvee_{\mathbf{B} \in \mathcal{B}(H)} \mathbf{B}\), that is, \(I \models H \leftrightarrow \bigvee_{\mathbf{B} \in \mathcal{B}(H)} \mathbf{B}\).
    Hence, \(I_=\) is a model of \(\comp(\prog)\), and by Lemma~\ref{lem:decl-neg-supp}, this is equivalent to \(T_\prog(I) = I\).
\end{proof}