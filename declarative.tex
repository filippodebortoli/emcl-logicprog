\chapter{Declarative interpretation}

\section{Definitions}

\begin{dfn}[Algebra]
    An \emph{algebra} (or \emph{pre-interpretation}) \(J\) for \gls{funcs} consists of:
    \begin{enumerate}
        \item A non-empty \emph{domain} \(D\),
        \item for all \(f \in \gls{funcs}^{(n)}\) and \(n \ge 0\), a mapping \(f_J \colon D^n \to D\). 
    \end{enumerate}
    A \emph{state} \(\sigma\) over \(D\) is a mapping \(\sigma \colon \gls{vars} \to D\).
    The state \(\sigma\) is extended to \gls{TU} in the following way: \(\sigma \colon \gls{TU} \to D\) such that \(\sigma(f(t_1,\dotsc,t_n)) = f_J(\sigma(t_1),\dotsc,\sigma(t_n))\) for every \(f \in \gls{funcs}^{(n)}\).
\end{dfn}

\begin{dfn}
    An \emph{interpretation} \(I\) for \gls{funcs} and \gls{preds} consists of an algebra \(J\) for \gls{funcs} (with domain \(D\)), together with a relation \(p_I \subseteq D^n\) for all \(p \in \gls{preds}^{(n)}\) and \(n \ge 0\).
\end{dfn}

\begin{dfn}[Herbrand stuff]
    Given \gls{funcs} and \gls{preds}:
    \begin{enumerate}
        \item the \emph{Herbrand universe} \gls{HU} is equal to \(TU_{\gls{funcs},\emptyset}\),
        \item the \emph{Herbrand base} \gls{HB} is equal to \(TB_{\gls{preds},\gls{funcs},\emptyset}\).
    \end{enumerate}
\end{dfn}

\begin{dfn}[Expressions, truth and models]
    An \emph{expression} is an atom, a query, a clause or a resultant.
    An expression \(E\) is \emph{true in \(I\) under \(\sigma\)}, in symbols \(I \models_{\sigma} E\), according to the following schema:
    \begin{enumerate}
        \item \(I \models_{\sigma} p(t_1,\dotsc,t_n)\) iff \(\sigma(t_1),\dotsc,\sigma(t_n)) \in p_I\),
        \item \(I \models_\sigma A_1,\dotsc,A_n\) iff \(I \models_\sigma A_i\) for \(i = \ito{n}\),
        \item \(I \models_\sigma A \from \mathbf{B}\) iff \(I \models_\sigma \mathbf{B}\) implies \(I \models_\sigma A\),
        \item \(I \models_\sigma \mathbf{A} \from \mathbf{B}\) iff \(I \models_\sigma \mathbf{B}\) implies \(I \models_\sigma \mathbf{A}\),
    \end{enumerate}
    If \(x_1,\dotsc,x_k\) are the variables occurring in \(E\), its \emph{existential closure} \(\exists{E}\) is \(\exists{x_1},\dotsc,\exists{x_k}E\), while the \emph{universal closure} \(\forall{E}\) is \(\forall{x_1},\dotsc,\forall{x_k}E\); moreover:
    \begin{enumerate}
        \item \(I \models \exists{E}\) iff \(I \models_\sigma E\) holds for some state \(\sigma\),
        \item \(I \models \forall{E}\) iff \(I \models_\sigma E\) holds for every state \(\sigma\),
        \item \(I\) is a \emph{model} of \(E\), in symbols \(I \models E\), iff \(I \models \forall{E}\).
    \end{enumerate}
\end{dfn}

\begin{dfn}[Consequence]
    Given sets of expressions \(S\), \(T\) and an interpretation \(I\), \(I\) is a \emph{model} of \(S\) iff \(I \models E\) for all \(E \in S\).
    The set \(T\) is a \emph{logical consequence} of \(S\), in symbols \(S \models T\), iff every model of \(S\) is a model of \(T\).
\end{dfn}

\begin{dfn}[Correct substitutions]
    For a program \prog, a query \(Q_0\) and a substitution \(\theta\),
    \begin{enumerate}
        \item \(\restr{\theta}{\Var(Q_0)}\) is a \emph{correct answer substitution} of \(Q_0\) iff \(\prog \models Q_0\theta\);
        \item \(Q_0\theta\) is a \emph{correct instance} of \(Q_0\) iff \(\prog \models Q_0\theta\).
    \end{enumerate}
\end{dfn}

\begin{dfn}[Term models]
    The \emph{term algebra} \(J\) for \gls{funcs} is defined as follows:
    its domain is \gls{TU} and \(F_J(t_1,\dotsc,t_n) := f(t_1,\dotsc,t_n)\) for all \(f \in \gls{funcs}^{(n)}\).
    
    A \emph{term interpretation} \(I\) for \gls{funcs} and \gls{preds} consists of the term algebra for \gls{funcs} and the set \(I \subseteq \gls{TB}\) denoting the set of atoms mapped to \(\top\).

    The interpretation \(I\) is a \emph{term model} of a set \(S\) of expressions iff \(I\) is a term interpretation and \(I\) is a model of \(S\).
\end{dfn}

\begin{dfn}[Herbrand models]
    The \emph{Herbrand algebra} \(J\) for \gls{funcs} is defined as follows:
    its domain is \gls{HU} and \(F_J(t_1,\dotsc,t_n) := f(t_1,\dotsc,t_n)\) for all \(f \in \gls{funcs}^{(n)}\).
    
    A \emph{Herbrand interpretation} \(I\) for \gls{funcs} and \gls{preds} consists of the Herbrand algebra for \gls{funcs} and the set \(I \subseteq \gls{HB}\) denoting the set of ground atoms mapped to \(\top\).

    The interpretation \(I\) is a \emph{Herbrand model} of a set \(S\) of expressions iff \(I\) is a Herbrand interpretation and \(I\) is a model of \(S\).
    A \emph{least Herbrand model} \(I\) for a set \(S\) of expressions satisfies \(I \subseteq I^\prime\) for all Herbrand models \(I^\prime\) of \(S\).
\end{dfn}

\begin{dfn}[Implication tree]
    An \emph{implication tree} wrt a program \prog is a finite tree which nodes are atoms and:
    \begin{enumerate}
        \item if \(A\) is a node with direct descendants \(B_1,\dotsc,B_n\), then \(A \from B_1,\dotsc,B_n \in \inst(P)\),
        \item if \(A\) is a leaf, then \(A \in \inst(P)\)
    \end{enumerate}
    where, for an expression \(E\) and a set of expressions \(S\), \(\inst(E)\) is the set of all instances of \(E\), \(\inst(S)\) is the set of all instances of all the elements of \(S\); \(\grnd(E)\) is the set of ground instances of \(E\) and \(\grnd(S)\) is the set of all ground instances of all the elements of \(S\).
\end{dfn}

\begin{dfn}[\(n\)-depth]
    A query \(Q\) is \emph{\(n\)-deep} if every atom in \(Q\) is the root of an implication tree and \(n\) is the total number of nodes in these trees.
\end{dfn}

\begin{dfn}[\(T_\prog\)]
    Given a program \prog and an interpretation \(I\),
    \begin{equation*}
        T_\prog = \lbrace A \mid A \from B_1,\dotsc,B_n \in \grnd(\prog), \lbrace B_1,\dotsc,B_n \rbrace \subseteq I \rbrace.
    \end{equation*}
\end{dfn}

\begin{dfn}[Success set]
    The \emph{success set} of a program \prog is
    \begin{equation*}
        \lbrace A \mid A\;\text{ground atom and there is a successful \gls{sld}-derivation of \(\prog \cup \lbrace A \rbrace\)}\rbrace.
    \end{equation*}
\end{dfn}
\section{Algorithms}

\section{Theorems \& friends}

\begin{lem}
    \label{decl:lem-1}
    Let \(Q \sldstep{c}{\theta} Q^\prime\) be an \gls{sld}-derivation step and \(Q\theta \from Q^\prime\) the resultant associated with it.
    Then, \(c \models Q\theta \from Q^\prime\).
\end{lem}
\begin{proof}
    Let \(Q = \mathbf{A}, B, \mathbf{C}\) with selected atom \(B\). Let \(H \from \mathbf{B}\) be the input clause and \(Q^\prime = (\mathbf{A},\mathbf{B},\mathbf{C})\theta\).

    Since \(H \from \mathbf{B}\) is a variant of \(c\), it follows that \(c \models H \form \mathbf{B}\).
    Being \(H\theta \from \mathbf{B}\theta\) an instance of the input clause, \(c \models H\theta \from \mathbf{B}\theta\).
    The substitution \(\theta\) is a unifier for \(H\) and \(B\), hence \(c \models B\theta \from \mathbf{B}\theta\).
    The rest of the context remains unchanged, therefore we can conclude that \(c \models Q\theta \from Q^\prime\).
\end{proof}

\begin{lem}
    \label{decl:lem-2}
    Let \(\xi\) be an \gls{sld}-derivation of \(\prog \cup \lbrace Q_0 \rbrace\).
    For \(i \ge 0\), let \(R_i\) be the resultant of level \(i\) of \(\xi\).
    Then, \(P \models R_i\).
\end{lem}
\begin{proof}
    By induction on the level \(i \ge 0\).
    \begin{description}
        \item[I.B. \((i = 0)\)] The implication \(R_0 \colon Q_0 \from Q_0\) holds trivially, hence \(P \models R_0\).
        \item[I.B. \((i = 1)\)] We know that \(R_1 \colon Q_0\theta_1 \from Q_1\). Using Lemma~\ref{decl:lem-1}, we obtain that \(P \models R_1\).
        \item[I.H.] Assume that \(P \models R_i\).
        \item[I.C. \((i \to i+1)\)] Let \(Q_i\theta_{i+1} \from Q_{i+1}\) be the resultant associated with the \((i+1)\)-st derivation step of \(\xi\).
        Then, the resultant of level \(i+1\), \(R_{i+1} \colon Q_0\theta_1\dotsb\theta_{i+1} \from Q_{i+1}\) is a logical consequence of it. (Can we really say it?)

        Thanks to the I.H. \(\prog \models Q_0\theta_1\dotsb\theta_i \from Q_i\).
        From Lemma~\ref{decl:lem-1}, \(\prog \models Q_i\theta_{i+1} \from Q_{i+1}\).
        Since \(R_i\theta_{i+1} = Q_0\theta_1\dotsb\theta_{i+1} \from Q_i\theta_{i+1}\), this let us conclude that \(\prog \models R_{i+1}\). 
    \end{description}
\end{proof}

\begin{thm}[Soundness of \gls{sld}-resolution]
    \label{sld:sound}
    If there exists a successful \gls{sld}-derivation of \(\prog \cup \lbrace Q_0 \rbrace\) with \gls{cas} \(\theta\), then \(P \models Q_0\theta\).
\end{thm}
\begin{proof}
    Let \(\xi\) be a successful \gls{sld}-derivation.
    Then, the resultant of level \(n\) corresponds to \(R_n \colon Q_0\theta_1\dotsb\theta_n \from \square\).
    Thanks to Lemma~\ref{decl:lem-2}, this yields \(\prog \models Q_0\theta_1\dotsb\theta_n\), and \(Q_0\theta_1\dotsb\theta_n = Q_0\restr{(\theta_1\dotsb\theta_n)}{\Var(Q_0)} = Q_0\theta\), where \(\theta\) is the \gls{cas}.
\end{proof}

\begin{cor}[of Theorem~\ref{sld:sound}]
    If there exists a successful \gls{sld}-derivation of \(\prog \cup \lbrace Q_0 \rbrace\), then \(\prog \models \exists{Q_0}\).
\end{cor}
\begin{proof}
    From Theorem~\ref{sld:sound}, it follows that \(\prog \models Q_0\theta\) for some \gls{cas} \(\theta\).
    Given a model \(I\) of \prog, \(I \models Q_0\theta\) holds; consequently, it also holds \(I \models \forall{(Q_0\theta)}\) and finally, \(I \models \exists{Q_0}\).
    Therefore, \(\prog \models \exists{Q_0}\).
\end{proof}

\begin{lem}
    \label{decl:lem-3}
    Given a term interpretation \(I\), an atom \(A\) and a program \prog,
    \begin{itemize}
        \item \(I \models A\) iff \(\inst(A) \subseteq I\),
        \item \(I \models \prog\) iff for each \(A \from B_1, \dotsc, B_n \in \inst(\prog)\) if \(\lbrace B_1,\dotsc,B_n \rbrace \subseteq I\) then \(A \in I\).
    \end{itemize}
\end{lem}

\begin{lem}
    \label{decl:lem-4}
    The term interpretation
    \begin{equation*}
        \mathcal{C}(\prog) = \lbrace A \mid A\;\text{is the root of some implication tree wrt \prog}\,\rbrace
    \end{equation*}
    is a model of \prog.
\end{lem}

\begin{lem}
    Given a Herbrand interpretation \(I\), an atom \(A\) and a program \prog,
    \begin{itemize}
        \item \(I \models A\) iff \(\grnd(A) \subseteq I\),
        \item \(I \models \prog\) iff for each \(A \from B_1, \dotsc, B_n \in \grnd(\prog)\) if \(\lbrace B_1,\dotsc,B_n \rbrace \subseteq I\) then \(A \in I\).
    \end{itemize}
\end{lem}

\begin{lem}
    The Herbrand interpretation
    \begin{equation*}
        \mathcal{M}(\prog) = \lbrace A \mid A\;\text{is the root of some ground implication tree wrt \prog}\,\rbrace
    \end{equation*}
    is a model of \prog.
\end{lem}

\begin{lem}
    \label{decl:lem-5}
    Suppose \(Q\theta\) is \(n\)-deep for some \(n \ge 0\).
    Then for every selection rule \rulesel there exists a successful \gls{sld}-derivation of \(\prog \cup \lbrace Q \rbrace\) with \gls{cas} \(\eta\) such that \(Q\eta\) is more general than \(Q\theta\).
\end{lem}

\begin{thm}[Completeness of \gls{sld}-resolution]
    \label{sld:comp}
    Suppose that \(\theta\) is a correct answer substitution of \(Q\).
    Then, for every selection rule \rulesel there exists a successful \gls{sld}-derivation of \(\prog \cup \lbrace Q \rbrace\) with \gls{cas} \(\eta\) such that \(Q\eta\) is more general than \(Q\theta\).        
\end{thm}
\begin{proof}
    Let \(Q = A_1,\dotsc,A_m\).
    Let \(\theta\) be a correct answer substitution of \(Q\).
    This implies that \(\prog \models Q\theta\).
    Since \(\mathcal{C}(\prog) \models \prog\) as said in Lemma~\ref{decl:lem-4} yields \(\mathcal{C}(\prog) \models A_1\theta,\dotsc,A_m\theta\), which, in turns, by Lemma~\ref{decl:lem-3}, allows to say that \(\inst(A_i\theta) \subseteq \mathcal{C}(\prog)\) for every \(i = ito{m}\).
    In particular, \(A_i\theta \in \mathcal{C}(\prog)\) for all \(i = \ito{m}\), which means that \(A_1\theta,\dotsc,A_m\theta\) is \(n\)-deep for some \(n \ge 0\).
    Using Lemma~\ref{decl:lem-5}, we can conclude that our original claim holds.
\end{proof}

\begin{cor}[of Theorem~\ref{sld:comp}]
    Suppose that \(\prog \models \exists{Q}\).
    Then, there exists a successful \gls{sld}-derivation of \(\prog \cup \lbrace Q \rbrace\).
\end{cor}
\begin{proof}
    Due to \(\prog \models \exists{Q}\) implying that \(\prog \models Q\theta\) for some substitution \(\theta\), we can say that \(\theta\) is a correct answer substitution of \(Q\).
    By Theorem~\ref{sld:comp}, this yields the result.
\end{proof}

\begin{thm}[Least Herbrand model]
    \(\mathcal{M}(\prog)\) is the least Herbrand model of \prog.    
\end{thm}
\begin{proof}
    Let \(I\) be a Herbrand model of \prog, and let \(a \in \mathcal{M}(\prog)\).
    By induction on \(i\), number of nodes in the ground implication tree wrt \prog with root \(A\).
    \begin{description}
        \item[I.B. \((i = 1)\)] Since \(A\) is the only node, it is a leaf, so \(A \in \grnd(\prog)\).
        This implies that \(I \models A\) (since \(I \models \prog\), hence \(A \in I\).
        \item[I.H.] Assume that the claim holds for \(i \ge 1\).
        \item[I.C. (\(i \to i+1\))]
        Let \(B_1,\dotsc,B_n\) be the direct descendants of \(A\).
        By induction hypothesis on the subtrees which roots are \(B_1,\dotsc,B_n\), \(B_1,\dotsc,B_n \in I\); moreover, \(A \from B_1,\dotsc,B_n \in \grnd(\prog)\).
        From \(B_1,\dotsc,B_n \in I\) we deduce that \(I \models A\) must hold, hence \(A \in I\).
    \end{description}
\end{proof}

\begin{thm}
    For every ground atom \(A\), \(\prog \models A\) iff \(\mathcal{M}(\prog) \models A\).
\end{thm}
\begin{proof}
    \begin{description}
        \item[``\(\to\)''] Since \(\prog \models A\) and \(\mathcal{M}(\prog) \models \prog\), it follows that \(\mathcal{M}(\prog) \models A\).
        \item[``\(\from\)''] Let
        \begin{equation*}
            I_H := \lbrace A \mid A\;\text{is a ground atom and \(I \models A\)}\,\rbrace
        \end{equation*}
        be a Herbrand interpretation.
        Assume that \(I \models \prog\).
        Then, \(I \models B \from B_1,\dotsc,B_n\) for all \(B \from B_1,\dotsc,B_n \in \grnd(\prog)\).
        This means that if \(I \models B_1\),\ldots,\(I \models B_n\) --- that is, \(B_1,\dotsc,B_n \in I_H\) --- then \(I \models B\), hence \(B \in I_H\).

        Therefore, \(I_H \models P\); from this, having assumed that \(A \in \mathcal{M}(\prog)\) and being it the least Herbrand model, we can conclude that \(A \in I_H\) and \(I \models A\).
    \end{description}
\end{proof}

\begin{thm}
    If \(T\) is a continuous operator on a CPO, then \(T^\omega\) exists and is the least (pre-)fixpoint of \(T\).
\end{thm}

\begin{lem}
    The operator \(T_\prog\) is finitary and monotonic.
\end{lem}

\begin{lem}
    A Herbrand interpretation \(I\) is a model of \prog iff \(T_\prog(I) \subseteq I\).
\end{lem}
\begin{proof}
    \(I \models \prog\) iff for every \(A \from B_1,\dotsc,B_n \in \grnd(\prog)\), \(\lbrace B_1,\dotsc,B_n \rbrace \subseteq I\) implies \(A \in I\).
    This is equivalent to say that, for every ground atom \(A\), \(A \in T_\prog(I)\) implies \(A \in I\), that is, \(T_\prog(I) \subseteq I\).
\end{proof}

\begin{thm}
    \(\mathcal{M}(\prog)\) is the fixpoint of \(T_\prog\) and equals 
    \begin{equation*}
        T_\prog^\omega = \lbrace A \mid A\;\text{ground atom}\;,\prog\models A \rbrace.
    \end{equation*}
\end{thm}

\begin{thm}
    For a ground atom \(A\), the following are equivalent:
    \begin{enumerate}
        \item \(\mathcal{M}(\prog) \models A\),
        \item \(\prog \models A\),
        \item every \gls{sld}-tree for \(\prog \cup \lbrace A \rbrace\) is successful,
        \item \(A\) is in the success set of \(\prog\).
    \end{enumerate}
\end{thm}
