% arara: xelatex
% arara: makeglossaries
% arara: xelatex

\documentclass{report}

\usepackage[T1]{fontenc}
\usepackage[english]{babel}

\usepackage{glossaries}

\newglossary{symbols}{sym}{sbl}{Notation}
\makeglossary

\newglossaryentry{alph}
{
    type=symbols,
    name={\ensuremath{\Sigma}},
    description={ranked alphabet}
}
\newglossaryentry{vars}
{
    type=symbols,
    name={\ensuremath{\mathcal{V}}},
    description={variables}
}
\newglossaryentry{funcs}
{
    type=symbols,
    name={\ensuremath{\mathcal{F}}},
    description={function symbols}
}
\newglossaryentry{TU}
{
    type=symbols,
    name={\ensuremath{TU_{\mathcal{F},\mathcal{V}}}},
    description={term universe}
}
\newglossaryentry{terms}
{
    type=symbols,
    name={\ensuremath{\mathcal{T}}},
    description={terms}
}
\newglossaryentry{tVar}
{
    type=symbols,
    name={\ensuremath{\mathrm{Var}(t)}},
    description={variables appearing in \(t\)}
}
\newglossaryentry{dom}
{
    type=symbols,
    name={\ensuremath{\mathrm{Dom}(\theta)}},
    description={domain of the substitution \(\theta\)}
}

\setacronymstyle{long-short}
\newacronym{mgu}{MGU}{most general unifier}

\usepackage{amsmath}
\usepackage{amsthm}
\usepackage{amssymb}
\usepackage[nude]{12many}
\usepackage{algorithm}
\usepackage{algorithm}
\usepackage[noend]{algpseudocode}

\newcommand*\Let[2]{\State #1 $\gets$ #2}
%\algrenewcommand\alglinenumber[1]{
%    {\sf\footnotesize\addfontfeatures{Colour=888888,Numbers=Monospaced}#1}}
%\algrenewcommand\algorithmicrequire{\textbf{Precondition:}}
%\algrenewcommand\algorithmicensure{\textbf{Postcondition:}}

\theoremstyle{definition}
\newtheorem{dfn}{Definition}[section]

\theoremstyle{plain}
\newtheorem{lem}{Lemma}[section]
\newtheorem{thm}{Theorem}[section]
\newtheorem{prp}{Proposition}[section]
\newtheorem{cor}{Corollary}[section]

\theoremstyle{remark}
\newtheorem{rem}{Remark}

\newcommand\restr[2]{{% we make the whole thing an ordinary symbol
  \left.\kern-\nulldelimiterspace % automatically resize the bar with \right
  #1 % the function
  \vphantom{\big|} % pretend it's a little taller at normal size
  \right|_{#2} % this is the delimiter
}}

% New definitions
\algnewcommand\algorithmicswitch{\textbf{switch}}
\algnewcommand\algorithmiccase{\textbf{case}}
\algnewcommand\algorithmicassert{\texttt{assert}}
\algnewcommand\Assert[1]{\State \algorithmicassert(#1)}%
% New "environments"
\algdef{SE}[SWITCH]{Switch}{EndSwitch}[1]{\algorithmicswitch\ #1\ \algorithmicdo}{\algorithmicend\ \algorithmicswitch}%
\algdef{SE}[CASE]{Case}{EndCase}[1]{\algorithmiccase\ #1}{\algorithmicend\ \algorithmiccase}%
\algtext*{EndSwitch}%
\algtext*{EndCase}%

\begin{document}

\section{Unification}

\subsection{Definitions}

\begin{dfn}[Ranked alphabet]
  A \emph{ranked alphabet} is a finite set \gls{alph} of symbols, each one with its \emph{arity}; \(\gls{alph}^{(n)}\) is the subset of \gls{alph} of \(n\)-ary symbols.
\end{dfn}

\begin{dfn}[Term universe]
  Given a set of \emph{variables} \gls{vars} and a ranked alphabet of \emph{function} symbols \gls{funcs}, the \emph{term universe} \gls{TU} is the smallest set \gls{terms} of \emph{terms} with:
  \begin{enumerate}
      \item \(\gls{vars} \subseteq \gls{terms}\),
      \item if \(f \in \gls{funcs}^{(0)}\) is a \emph{constant} symbol, then \(f \in \gls{terms}\),
      \item if \(f \in \gls{funcs}^{(n)}\), with \(n \ge 1\) and \(t_1,\dotsc,t_n \in \gls{terms}\), then \(f(t_1,\dotsc,t_n) \in \gls{terms}\).
  \end{enumerate}
  A term \(t\) is \emph{ground} if \(\gls{tVar} = \emptyset\).
  A term \(s\) is a \emph{sub-term} of \(t\) if it is a sub-string of \(t\).
\end{dfn}

\begin{dfn}[Substitution]
    Given a finite set \(X \subseteq \gls{vars}\), a \emph{substitution} is a function \(\theta \colon X \to \gls{TU}\) such that \(x \ne \theta(x)\) for every \(x \in X\).

    If \(X = \lbrace x_1,\dotsc,x_n \rbrace\) and \(\theta(x_i) = t_i\), the notation is \(\theta = \lbrace x_1/t_1,\dotsc,x_n/t_n \rbrace\).
    The \emph{domain} of \(\theta\) is \(\gls{dom} := \lbrace x_1,\dotsc,x_n \rbrace\).
    \begin{enumerate}
        \item if \(X = \emptyset\), \(\theta = \epsilon\) is the \emph{empty} substitution;
        \item a substitution is \emph{ground} if \(t_1,\dotsc,t_n\) are ground terms;
        \item a \emph{pure} variable substitution replace variables with variables;
        \item a substitution is a \emph{renaming} if \(\lbrace t_1,\dotsc,t_n \rbrace = \lbrace x_1,\dotsc,x_n \rbrace\);
        \item for \(Y \subseteq \gls{vars}\), \(\restr{\theta}{Y} := \lbrace y/t \mid y/t \in \theta, y \in Y \rbrace \).
    \end{enumerate} 
\end{dfn}

\begin{dfn}[Terms and substitutions]
    If \(x \in \gls{vars}\), then
    \begin{equation*}
        x\theta := \begin{cases}
            \theta(x) & x \in \gls{dom} \\
            x & x \notin \gls{dom}
        \end{cases}
    \end{equation*}
    Otherwise, \(f(t_1,\dotsc,t_n)\theta := f(t_1\theta,\dotsc,t_n\theta)\).

    The term \(t\) is an \emph{instance} of \(s\) if there is a substitution \(\theta\) with \(s\theta = t\); in this case, \(s\) is \emph{more general} than \(t\).
    The term \(t\) is a \emph{variant} of \(s\) if \(t\) is an instance of \(s\) wrt \(\theta\) and \(\theta\) is a renaming.
\end{dfn}

\begin{dfn}[Composition]
    Given substitutions \(\theta\) and \(\eta\), their \emph{composition} is defined as \((\theta\eta)(x) := (x\theta)\eta\) for \(x \in \gls{vars}\).
\end{dfn}

\begin{dfn}[Substitution ordering]
    Given substitutions \(\theta\) and \(\eta\), \(\theta\) is \emph{more general} than \(\eta\) iff \(\eta = \theta\tau\) for some substitution \(\tau\).
\end{dfn}

\begin{dfn}[Unifiers]
    A substitution \(\theta\) is a unifier for \(s,t \in \gls{terms}\) iff \(s\theta = t\theta\); in this case, \(s\) and \(t\) are \emph{unifiable}.

    A unifier \(\theta\) for \(s\) and \(t\) is a \gls{mgu} of \(s\) and \(t\) iff it is more general than all the unifiers of \(s\), \(t\).

    A unifier \(\theta\) for the \emph{unification problem} \(E = \lbrace s_1 \doteq t_1,\dotsc, s_n \doteq t_n \rbrace\) satisfies \(s_i\theta = t_i\theta\) for all \(i = \ito{n}\), and it is the \gls{mgu} of \(E\) if it is more general than all the unifiers of \(E\).
\end{dfn}

\begin{dfn}[Unification problems]
    A \emph{unification problem} is a set \(E = \lbrace s_1 \doteq t_1,\dotsc, s_n \doteq t_n \rbrace\) where \(s_i,t_i \in \gls{terms}\) for all \(i = \ito{n}\).
    The unification problems \(E\) and \(E^\prime\) are \emph{equivalent} if they have the same set of unifiers.

    A set \(\lbrace x_i \doteq t_1,\dotsc,x_n \doteq t_n \rbrace\) is \emph{solved} if \(x_i \ne x_j\) for \(i \ne j\) and no \(x_i\) occurs in \(t_j\), \(i,j = \ito{n}\).
\end{dfn}

\subsection{Algorithms}

\begin{algorithm}
    \caption{Martelli-Montanari: computing \gls{mgu} for a unification problem \label{alg:mm}}
    \begin{algorithmic}[1]
      \Require{\(E\) is a set of pair of terms}
      \Statex
      \Repeat
      \State Choose nondeterministically a pair \(t \doteq s\)
      \Switch{\(t \doteq s\)}
      \Case{\(f(s_1,\dotsc,s_n) \doteq f(t_1,\dotsc,t_n)\)}
      \State replace by \(s_1 \doteq t_1,\dotsc,s_n \doteq t_n\) \label{mm:1}
      \EndCase
      \Case{\(f(s_1,\dotsc,s_n) \doteq g(t_1,\dotsc,t_m)\) where \(f \ne g\)}
      \State halt with failure \Comment{\emph{clash} failure} \label{mm:2}
      \EndCase
      \Case{\(x \doteq x\)}
      \State delete the pair \label{mm:3}
      \EndCase
      \Case{\(t \doteq x\), \(t \notin \gls{vars}\)}
      \State replace by \(x \doteq t\) \label{mm:4}
      \EndCase
      \Case{\(x \doteq t\), \(x \notin \gls{tVar}\) and \(x\) occurring in some other pair}
      \State perform substitution \(\lbrace x/t \rbrace\) on all other pairs \label{mm:5}
      \EndCase
      \Case{\(x \doteq t\), \(x \in \gls{tVar}\) and \(x \ne t\)}
      \State halt with failure \Comment{\emph{occur check} failure} \label{mm:6}
      \EndCase
      \EndSwitch
      \Until{No action can be performed}
      \State \Return{success}
    \end{algorithmic}
  \end{algorithm}

  \begin{rem}[Implementation of Martelli-Montanari]
      In most Prolog implementations, the occur check performed in~\algref{alg:mm}{mm:6} does not apply, for the sake of efficiency.
      This means that the algorithm terminates with success in the case \(\lbrace x \doteq f(x) \rbrace\), despite the terms not being unifiable.

      Similarly,~\algref{alg:mm}{mm:5} is usually not implemented in Prolog systems.
      This may lead to sets where the \gls{mgu} is implicitly represented, as in \(\lbrace x \doteq f(y), y \doteq g(a) \rbrace\).
  \end{rem}

\subsection{Theorems \& friends}

\begin{lem}[Variants and instances]
    The term \(t\) is a variant of \(s\) iff \(t\) is an instance of \(s\) and \(s\) is an instance of \(t\).
\end{lem}

\begin{lem}[How-to build compositions]
    Let \(\theta = \lbrace x_1/t_1,\dotsc,x_n/t_n \rbrace\) and \(\eta = \lbrace y_1/s_1,\dotsc,y_m/s_m \rbrace\).
    The composition \(\theta\eta\) is obtained by applying the following steps to the sequence:
    \begin{equation*}
        x_1/t_1\eta,\dotsc,x_n/t_n\eta,y_1/s_1,\dotsc,y_m/s_m
    \end{equation*}
    \begin{enumerate}
        \item Remove all bindings \(x_i/t_i\eta\) where \(x_i = t_i\eta\) and all bindings \(y_j/s_j\) where \(y_j \in \lbrace x_1,\dotsc,x_n \rbrace\),
        \item Form a substitution from the resulting sequence.
    \end{enumerate}
\end{lem}

\begin{lem}[Unification problem and \gls{mgu}s]
    \label{lem:uni-mgu}
    If \(E = \lbrace s_1 \doteq t_1,\dotsc, s_n \doteq t_n \rbrace\) is solves, then \(\theta = \lbrace x_1/t_1,\dotsc,x_n/t_n \rbrace\) is an \gls{mgu} of \(E\).    
\end{lem}
\begin{proof}
    Notice that \(x_i\theta = t_i = t_i\theta\) holds for all \(i = \ito{n}\), since \(E\) is solved.
    Then, for every unifier \(\eta\) of \(E\), \(x_i\eta = t_i\eta = x_i\theta\eta\) for all \(i = \ito{n}\), while \(x\eta = x\theta\eta\) if \(x \notin \lbrace x_1,\dotsc,x_n \rbrace\). Hence, \(\eta = \theta \eta\).
\end{proof}

\begin{thm}[Martelli-Montanari]
    If a unification problem \(E\) has a unifier, then the algorithm~\ref{alg:mm} successfully terminates and produces a solved set \(E^\prime\) equivalent to \(E\); otherwise, it terminates with failure.
\end{thm}
\begin{proof}
    First, we prove termination.
    Define a variable \(x\) as \emph{solved} in E iff \(x \doteq t \in E\) and that is the unique occurrence of \(x\) in \(E\).
    Define the following:
    \begin{itemize}
        \item \(uns(E)\) is the number of unsolved variables in \(E\),
        \item \(lfun(E)\) is the number of occurrences of function symbols in the first component of pairs of \(E\),
        \item \(card(E)\) is the number of pairs in \(E\).
    \end{itemize}
    Then, each successful action of the Martelli-Montanari algorithm decreases the triplet \((uns(E),lfun(E),card(E))\) wrt \(\prec_3\) (the lexicographic ordering on \(\mathbb{N}^3\):
    \begin{itemize}
        \item if~\algref{alg:mm}{mm:1} is applied, then \((u,l,c) \succ_3 (u - k, l - 1, c + n - 1)\) where \(0 \le k \le n\) is the number of \emph{new} solved variables obtained;
        \item if~\algref{alg:mm}{mm:3} is applied, then \((u,l,c) \succ_3 (u - k, l, c - 1)\) where \(k = 1\) if \(x\) becomes solved, \(0\) otherwise;
        \item if~\algref{alg:mm}{mm:4} is applied, then \((u,l,c) \succ_3 (u - k, l - m, c)\) where \(k = 1\) if \(x\) becomes solved, \(0\) otherwise and \(m\) is the number of function symbols appearing in \(t\);
        \item if~\algref{alg:mm}{mm:5} is applied, then \((u,l,c) \succ_3 (u - 1, l + k, c)\), since \(x\) becomes solved and some \(k\) function symbols in \(t\) may appear in the lhs of a pair.
    \end{itemize}
    Since \((\mathbb{N}^3,\prec_3)\) is a well-founded order, termination is guaranteed.

    Next, we prove that each successful action replaces the set of pairs with an equivalent one. For~\algref{alg:mm}{mm:1}, \algref{alg:mm}{mm:3} and \algref{alg:mm}{mm:4} this is trivially true.
    Regarding~\algref{alg:mm}{mm:5}, let \(\theta\) be a unifier of \(E \cup \lbrace x \doteq t \rbrace\). Then, \(\theta\) is a unifier of \(E\) and \(x\theta = t\theta\).
    This means that, by applying the substitution \(\lbrace x/t \rbrace\) to \(E\), \(\theta\) is still a unifier for \(E\lbrace x/t \rbrace\).
    Therefore, \(\theta\) is a unifier for \(E\lbrace x/t \rbrace \cup \lbrace x \doteq t \rbrace\).

    Moving on, if the algorithm successfully terminates:
    \begin{itemize}
        \item All the actions have been exhaustively applied, so each pair in \(E\) is of the form \(x \doteq t\) with \(x \in \gls{vars}\);
        \item Using a similar argument, the variables in the first component of all the pairs in \(E\) are pairwise disjoint and they do not occur in any second component of a pair in \(E\).
    \end{itemize}
    Hence, if the algorithm succesfully terminates, the obtained set is solved.

    If the algorithm instead halts with failure, the obtained set of pairs does not have a unifier, since one of the following holds:
    \begin{itemize}
        \item \(f(s_1,\dotsc,s_n) \doteq g(t_1,\dotsc,t_m)\) with \(f \ne g\) occurs in \(E\), and these two terms are not unifiable;
        \item \(x \doteq t\) where \(x\) is a proper sub-term of \(t\) occurs in \(E\) and these terms are not unifiable.
    \end{itemize}
    This consideration concludes the proof.
\end{proof}

\begin{cor}[MM and \gls{mgu}]
    Lemma~\ref{lem:uni-mgu} implies that, if the algorithm~\ref{alg:mm} succeeds, \(E^\prime\) determines an \gls{mgu} of \(E\).
\end{cor}

\printglossaries
\end{document}