\chapter{Negation: Procedural int.}

\section{Definitions}

\begin{dfn}[Literal]
    A \emph{literal} is of the form \(A\) or \(\neg A\), where \(A\) is an atom.
    A \emph{ground literal} is derived from a ground atom. 
\end{dfn}

\begin{dfn}
    An \emph{extended query} is a finite sequence of literals.
    An \emph{extended clause} is of the form \(H \from \mathbf{B}\) where \(H\) is an atom and \(\mathbf{B}\) an extended query.
    An \emph{extended program} is a finite set of extended clauses.
\end{dfn}

\begin{dfn}[\gls{naf}]
    \begin{enumerate}
        \item Suppose \(\neg A\) is selected in the query \(Q = \mathbf{L}, \neg A, \mathbf{N}\);
        \item if \(\prog \cup \lbrace A \rbrace\) succeeds, then the derivation of \(\prog \cup \lbrace Q \rbrace\) fails at this point;
        \item if all derivations of \(\prog \cup \lbrace A \rbrace\) fail, then \(Q\) resolves to \(Q^\prime = \mathbf{L},\mathbf{N}\).
    \end{enumerate}
    We can say that \(\neg A\) succeeds iff \(A\) finitely fails, and \(\neg A\) finitely fails iff \(A\) succeeds.
\end{dfn}

\begin{dfn}
    In \gls{sldnf}, two cases are distinguished:
    \begin{enumerate}
        \item \(Q = \mathbf{L}, A, \mathbf{N}\) is the query, and the selected literal \(A\) is positive:
        \begin{itemize}
            \item \(H \from \mathbf{M}\) variant of a clause \(c\) which is variable-disjoint with \(Q\), \(\theta\) \gls{mgu} of \(A\) and \(H\),
            \item \(Q^\prime = (\mathbf{L},\mathbf{M},\mathbf{N})\theta\) \emph{\gls{sldnf}-resolvent} of \(Q\) and \(c\) wrt \(A\) with \(\theta\)
            \item the \gls{sldnf}-derivation step as \(Q \sldstep{c}{\theta} Q^\prime\).
        \end{itemize}
        \item \(Q = \mathbf{L}, \neg A, \mathbf{N}\) is the query, and the selected literal \(\neg A\) is negative and \emph{ground}:
        \begin{itemize}
            \item \(Q^\prime = \mathbf{L},\mathbf{N}\) is the \gls{sldnf}-resolvent of \(Q\) wrt \(\neg A\) with \(\epsilon\);
            \item the \gls{sldnf}-derivation step as \(Q \sldstep{\epsilon}{} Q^\prime\).
        \end{itemize}
    \end{enumerate}
\end{dfn}

\begin{dfn}
    A maximal sequence of \gls{sldnf}-derivation steps
    \begin{equation*}
        Q_0 \sldstep{c_1}{\theta_1} Q_1 \dotsb Q_n
        \sldstep{c_{n+1}}{\theta_{n+1}} Q_{n+1} \dotsb
    \end{equation*}
    is a \emph{pseudo-derivation} of \(\prog \cup \lbrace Q_0 \rbrace\) if:
    \begin{itemize}
        \item \(Q_0,\dotsc,Q_{n+1},\dotsc\) are either empty queries or have one selected literal in it;
        \item \(\theta_1,\dotsc,\theta_{n+1},\dotsc\) are substitutions;
        \item \(c_1,\dotsc,c_{n+1},\dotsc\) are all clauses of \(\prog\), in case a positive literal has been selected in the preceding query;
        \item for every \gls{sldnf}-derivation step with input clause, standardization apart holds.
    \end{itemize}
\end{dfn}

\begin{dfn}[Forests]
    \label{dfn:forest}
    A \emph{forest} is a triple \(F = (\mathcal{T},T,\subs)\) where:
    \begin{enumerate}
        \item \(\mathcal{T}\) is a set of trees where
        \begin{itemize}
            \item nodes are queries,
            \item a literal is selected in each non-empty query,
            \item leaves may be marked as \emph{success}, \emph{failure} or \emph{floundered}.
        \end{itemize}
        \item \(T \in \mathcal{T}\) is the main tree;
        \item \(\subs\) assigns to some nodes of trees in \(\mathcal{T}\) with selected negative ground literal \(\neg A\) a subsidiary tree of \(\mathcal{T}\) with root \(A\).
    \end{enumerate}
    The tree \(T\) is \emph{successful} iff it contains a leaf marked as \emph{success}; it is \emph{finitely failed}, iff it is finite and all leaves are marked as \emph{failure}.
\end{dfn}

\begin{dfn}[pre-\gls{sldnf}-trees]
    The class of \emph{pre-\gls{sldnf}-trees} for a program \prog is the smallest class \(\mathcal{C}\) of forests such that:
    \begin{enumerate}
        \item for every query \(Q\), the initial pre-\gls{sldnf}-tree \((\lbrace T_Q \rbrace, T_Q, \subs)\) is in \(\mathcal{C}\), where \(T_Q\) contains the single node \(Q\) and \(\subs(Q)\) is undefined;
        \item for every \(F \in \mathcal{C}\), the extension of \(F\) is in \(\mathcal{C}\).
    \end{enumerate}
\end{dfn}

\begin{dfn}[\gls{sldnf}-trees]
    A \gls{sldnf}-tree is a limit of a sequence \(F_0,F_1,\dotsc\), where \(F_0\) is the initial pre-\gls{sldnf}-tree and \(F_{i+1}\) is the extension of \(F_i\), for \(i \in \mathbb{N}\).

    A \gls{sldnf}-tree for \(\prog \cup \lbrace Q \rbrace\) is a \gls{sldnf}-tree in which \(Q\) is the root of the main tree.

    A \gls{sldnf}-tree is \emph{successful} (resp. \emph{finitely failed}) iff its main tree is.
    A \gls{sldnf}-tree is \emph{finite} iff no infinite path of nodes \(N_0,N_1,\dotsc\) exists in it, such that, for all \(i \in \mathbb{N}\), either \(N_{i+1}\) is a direct descendant of \(N_i\) or it is the root of \(\subs(N_i)\).
\end{dfn}

\begin{dfn}[\gls{sldnf}-derivation]
    A \emph{\gls{sldnf}-derivation} of \(\prog \cup \lbrace Q \rbrace\) is a branch in the main tree of a \gls{sldnf}-tree \(F\), for \(\prog \cup \lbrace Q \rbrace\) together with the set of all trees in \(F\) which roots can be reached from the nodes in the branch.

    A \gls{sldnf}-derivation is \emph{successful} iff it ends with \(square\).
    In that case, given that the main tree of an \gls{sldnf}-tree for \(\prog \cup \lbrace Q_0 \rbrace\) contains a branch
    \begin{equation*}
        \xi = Q_0 \sldstep{}{\theta_1} Q_1 \dotsb Q_{n-1}
        \sldstep{}{\theta_n} \square
    \end{equation*}
    the \gls{cas} of \(Q_0\) wrt \(\xi\) is \(\restr{(\theta_1\dotsb\theta_n)}{\Var(Q_0)}\).
\end{dfn}

\begin{dfn}[Extended selection rule]
    An \emph{extended selection rule} is a function \rulesel which, given a pre-\gls{sldnf}-tree \(F = (\mathcal{T},T,\subs)\), selects a literal in every non-empty, unmarked leaf in every tree in \(\mathcal{T}\).

    A \gls{sldnf}-tree \(F\) is \emph{according to} \rulesel iff it is the limit of a sequence of pre-\gls{sldnf}-trees in which literals are selected according to \rulesel.
    
    The selection rule \rulesel is \emph{safe} if it never selects a non-ground negative literal.
\end{dfn}

\begin{dfn}[Allowed vs Blocked]
    A query \(Q\) is \emph{blocked}, when it is non-empty and contains exclusively non-ground negative literals.
    A query \(Q\) is \emph{allowed} if every variable occurring in \(Q\) occurs in one of its positive literals.
    A clause \(H \from \mathbf{B}\) is allowed if \(\neg H\) and \(\mathbf{B}\) are both allowed --- thus, a fact \(H\) is allowed iff it is a ground atom.
    A program is \emph{allowed} if all its clauses are allowed.

    \(\prog \cup \lbrace Q \rbrace\) \emph{flounders} if some \gls{sldnf}-tree for it contains a blocked node.
\end{dfn}

\begin{dfn}[Extended Prolog tree]
    Let \prog be an extended program and \(Q_0\) an extended query.
    The \emph{extended} Prolog tree for \(\prog \cup \lbrace Q_0 \rbrace\) is a forest of finitely branching, ordering trees of queries, possibly marked with \emph{success} or \emph{failure}, produced in the following way:
    \begin{enumerate}
        \item Start with the forest \((\lbrace T_{Q_0} \rbrace, T_{Q_0},\subs)\) where \(T_{Q_0}\) contains the single node \(Q_0\) and \(\subs(Q_0)\) is undefined;
        \item Repeatedly apply to the current forest \(F = (\mathcal{T},T,\subs)\) and leftmost unmarked leaf \(Q\) in \(T_1\), where \(T_1 \in \mathcal{T}\) is the leftmost, most nested subsidiary tree with an unmarked leaf, the operation \textbf{expand}\((F,Q)\).
    \end{enumerate}
\end{dfn}

\section{Algorithms}

\begin{dfn}[Extensions]
    An \emph{extension} of a forest \(F\) (as in Definition~\ref{dfn:forest}) is obtained in the following way:
    \begin{enumerate}
        \item Every occurrence of the empty query is marked as \emph{success};
        \item for every non-empty query \(Q\), which is an unmarked leaf in some tree in \(\mathcal{T}\), perform the following action, assuming that \(L\) is the selected literal of \(Q\):
        \begin{itemize}
            \item Assume that \(L\) is positive.
            If \(Q\) has no \gls{sldnf}-resolvents, then \(Q\) is marked as \emph{failure}.
            Otherwise, for every clause \(c\) which is applicable to \(L\), exactly one direct descendant of \(Q\) is added; this descendant is an \gls{sldnf}-resolvent of \(Q\) and \(c\) wrt \(L\).
            \item Assume that \(L = \neg A\) is negative.
            If \(A\) is non-ground, then \(Q\) is marked as \emph{floundered}.
            Otherwise, a case on \(\subs(Q)\) is applied:
            \begin{itemize}
                \item if \(\subs(Q)\) is undefined, a new tree \(T^\prime\) with a single node \(A\) is added to \(\mathcal{T}\) and \(\subs(Q)\) is set to \(T^\prime\);
                \item if \(\subs(Q)\) is defined and successful, then \(Q\) is marked as \emph{failure};
                \item if \(\subs(Q)\) is defined and finitely failed, the \gls{sldnf}-resolvent of \(Q\) is added as the only direct descendant of \(Q\);
                \item Otherwise, no action is applied.
            \end{itemize} 
        \end{itemize}
    \end{enumerate}
\end{dfn}

\paragraph{Prolog tree expansion.}
The operation \textbf{expand}\((F,Q)\) is defined by:
\begin{enumerate}
    \item if \(Q = \square\), mark \(Q\) with \emph{success} and, if \(T_1 \ne T\), remove from \(T_1\) all edges to the right of the branch that ends with \(Q\);
    \item if \(Q\) has no LDNF-resolvents, then mark \(Q\) with \emph{failure};
    \item let \(L\) be the leftmost literal in \(Q\):
    \begin{itemize}
        \item if \(L\) is positive, add for each clause that is applicable to \(L\) an LDNF-resolvent as descendant of \(Q\), such that the order of the clauses is respected;
        \item if \(L = \neg A\) is negative,
        \begin{itemize}
            \item if \(\subs(Q)\) is undefined, add a new tree \(T^\prime = A\) and set \(\subs(Q)\) to \(T^\prime\);
            \item if \(\subs(Q)\) is defined and successful, then mark \(Q\) with \emph{failure};
            \item if \(\subs(Q)\) is defined and finitely failed, then add in \(T_1\) the  \gls{sldnf}-resolvent of \(Q\) as the only direct descendant of \(Q\).
        \end{itemize}
    \end{itemize}
\end{enumerate}

\section{Theorems \& friends}

\begin{thm}
    \begin{enumerate}
        \item Every \gls{sldnf}-tree is the limit of a unique sequence of pre-\gls{sldnf}-trees.
        \item If the \gls{sldnf}-tree \(F\) is the limit of the sequence \(F_0,F_1,\dotsc\), then:
        \begin{enumerate}
            \item \(F\) is successful and yields \gls{cas} \(\theta\) iff some \(F_i\) is successful and yields \gls{cas} \(\theta\),
            \item \(F\) is finitely failed iff some \(F_i\) is finitely failed.
        \end{enumerate}
    \end{enumerate}
\end{thm}

\begin{thm}
    Suppose that \(\prog\) and \(Q\) are allowed.
    Then, \(\prog \cup \lbrace Q \rbrace\) does not flounder and if \(\theta\) is a \gls{cas} of \(Q\), then \(Q\theta\) is ground.
\end{thm}